\documentclass{article}
\usepackage{graphicx} % Required for inserting images
\title{April Fools' Day RFCs}
\author{Adam Lamers}
\date{March 2023}

\begin{document}

\maketitle
\tableofcontents

\section{Request for Comments}
    A Request for Comments (RFC), in the context of Internet governance, is a type of publication from the Internet Engineering Task Force (IETF) and the Internet Society (ISOC), usually describing methods, behaviors, research, or innovations applicable to the working of the Internet and Internet-connected systems \cite{wikipediaRFC}
\section{April Fools' Day RFC}
\subsection{The Twelve Networking Truths}
\subsubsection{Introduction}

   This Request for Comments (RFC) provides information about the
   fundamental truths underlying all networking. These truths apply to
   networking in general, and are not limited to TCP/IP, the Internet,
   or any other subset of the networking community.

\subsubsection{The Fundamental Truths} 
 \begin{enumerate}
    
    \item  It Has To Work.
    \item  No matter how hard you push and no matter what the priority,
    you can't increase the speed of light.
        
    \item (corollary). No matter how hard you try, you can't make a
             baby in much less than 9 months. Trying to speed this up
             \textit{*might*} make it slower, but it won't make it happen any
             quicker.
    
   \item With sufficient thrust, pigs fly just fine. However, this is
        not necessarily a good idea. It is hard to be sure where they
        are going to land, and it could be dangerous sitting under them
        as they fly overhead.

   \item  Some things in life can never be fully appreciated nor
        understood unless experienced firsthand. Some things in
        networking can never be fully understood by someone who neither
        builds commercial networking equipment nor runs an operational
        network.

    \item  It is always possible to aglutenate multiple separate problems
        into a single complex interdependent solution. In most cases
        this is a bad idea.

    \item  It is easier to move a problem around (for example, by moving
        the problem to a different part of the overall network
        architecture) than it is to solve it.
    \item(corollary). It is always possible to add another level of
             indirection.
    
    \item  It is always something
    \item (corollary). Good, Fast, Cheap: Pick any two (you can't  have all three).
    
    \item  It is more complicated than you think.

    \item  For all resources, whatever it is, you need more.

    \item (corollary) Every networking problem always takes longer to
            solve than it seems like it should.

    \item One size never fits all.

    \item Every old idea will be proposed again with a different name and
        a different presentation, regardless of whether it works.

    \item In protocol design, perfection has been reached not when there
        is nothing left to add, but when there is nothing left to take
        away. \cite{networkingTruthsAuthors}
       
\end{enumerate}
\subsection {The Address is the Message}
Declaring that the \textbf{address} is the \textbf{message}, the IPng WG has selected a
   packet format which includes 1696 bytes of address space.  This
   length is a multiple of 53 and is completely compatible with ATM
   architecture.  Observing that it's not what you know but who you
   know, the IPng focused on choosing an addressing scheme that makes it
   possible to talk to everyone while dispensing with the irrelevant
   overhead of actually having to say anything.

   \underline{Security experts hailed this as a major breakthrough}.  With no
   content left in the packets, all questions of confidentiality and
   integrity are moot.  Intelligence and law enforcement agencies
   immediately refocused their efforts to detect who's talking to whom,
   and are silently thankful they can avoid divisive public debate about
   key escrow, export control and related matters.

   Although the IPng WG declared there should be more than enough
   address space for everyone, service providers immediately began vying
   for reserved portions of the address space. \cite{theAddressIsTheMessageAuthor}

\subsection{Hyper Text Coffee Pot Control Protocol (HTCPCP/1.0)}

\begin{figure}[h]
    \centering
    \includegraphics[width=0.8\textwidth]{HtcpcpTeapot}
    \caption{Back-end infrastructure of error418.net, which implements HTCPCP}
    \label{fig:mesh1}
    \end{figure}

There is \textbf{\textit{\underline{coffee}}} all over the world. Increasingly, in a world in which
   computing is ubiquitous, the computists want to make \textbf{\textit{\underline{coffee}}}.\textbf{\textit{\underline{Coffee}}}
   brewing is an art, but the distributed intelligence of the web-
   connected world transcends art.  Thus, there is a strong, dark, rich
   requirement for a protocol designed espressoly for the brewing of
   \textbf{\textit{\underline{coffee}}}. \textbf{\textit{\underline{Coffee}}} is brewed using \textbf{\textit{\underline{coffee}}}  pots.  Networked \textbf{\textit{\underline{coffee}}} pots
   require a control protocol if they are to be controlled.

   Increasingly, home and consumer devices are being connected to the
   Internet. Early networking experiments demonstrated vending devices
   connected to the Internet for status monitoring [COKE]. One of the
   first remotely operated machine to be hooked up to the Internet,
   the Internet Toaster, (controlled via SNMP) was debuted in 1990
   [RFC2235].

   

   The demand for ubiquitous appliance connectivity that is causing the
   consumption of the IPv4 address space. Consumers want remote control
   of devices such as coffee pots so that they may wake up to freshly
   brewed \textbf{\textit{\underline{coffee}}}, or cause coffee to be prepared at a precise time after
   the completion of dinner preparations. \cite{coffeeAuthors}
   
\subsection{UTF-9 and UTF-18
              Efficient Transformation Formats of Unicode}
\begin{itemize}
    \item UTF-9 (Unicode Transformation Format-9) is a variable-length character encoding format. The highest bit of each nonet (i.e. 9-bit byte) is used as the continuation flag, the remaining octet is the ISO-10646 character code. For characters 0x0-0xFF it means full backward compatibility. All Unicode characters can be encoded in UTF-9, so there is no need to use U+D800 - U+DBFF surrogate codes
    \item UTF-18 (Unicode Transformation Format-18) is a fixed-length character encoding format. It uses two nonets. Characters in the range U+0000 - U+2FFFF remain unchanged, and in the range U+E0000 - U+EFFFFF are shifted by 0x70000, to the range 0x30000 - 0x3ffff. Other values ​​cannot be represented in UTF-18. \cite{UTF8UTF16Authors}
\end{itemize}
\begin{center}
 \begin{tabular}{| c | c | c | c||} 
 \hline
 UNICODE & UTF-9 & UTF-18 & UTF-8 \\ [0.5ex] 
 \hline\hline
 	U+0041 & 101 & 000101 & 101 \\ 
 \hline
  U+0104 & 401 004 & 000404 & 304-204 \\
 \hline
  U+AC00 & 654 000 &  126000 & 352 260 200 \\
 \hline
\end{tabular}
\end{center}
\subsection{A Compact Representation of IPv6 Addresses}
\begin{enumerate}

   \item Abstract IPv6 addresses, being 128 bits long, need 32 characters to write in
   the general case, if standard hex representation, is used, plus more
   for any punctuation inserted (typically about another 7 characters,
   or 39 characters total).  This document specifies a more compact
   representation of IPv6 addresses, which permits encoding in a mere 20
   bytes.

   \item Introduction

   It is always necessary to be able to write in characters the form of
   an address, though in actual use it is always carried in binary.  For
   IP version 4 (IP Classic) the well known dotted quad format is used.
   That is, 10.1.0.23 is one such address.  Each decimal integer
   represents a one octet of the 4 octet address, and consequently has a
   value between 0 and 255 (inclusive).  The written length of the
   address varies between 7 and 15 bytes.

   For IPv6 however, addresses are 16 octets long [IPv6], if the old
   standard form were to be used, addresses would be anywhere between 31
   and 63 bytes, which is, of course, untenable.

   Because of that, IPv6 had chosen to represent addresses using hex
   digits, and use only half as many punctuation characters, which will
   mean addresses of between 15 and 39 bytes, which is still quite long.
   Further, in an attempt to save more bytes, a special format was
   invented, in which a single run of zero octets can be dropped, the
   two adjacent punctuation characters indicate this has happened, the
   number of missing zeroes can be deduced from the fixed size of the
   address.

   In most cases, using genuine IPv6 addresses, one may expect the
   address as written to tend toward the upper limit of 39 octets, as
   long strings of zeroes are likely to be rare, and most of the other groups of 4 hex digits are likely to be longer than a single non-zero
   digit (just as MAC addresses typically have digits spread throughout
   their length).

   This document specifies a new encoding, which can always represent
   any IPv6 address in 20 octets.  While longer than the shortest
   possible representation of an IPv6 address, this is barely longer
   than half the longest representation, and will typically be shorter
   than the representation of most IPv6 addresses.

\item Current formats

   [AddrSpec] specifies that the preferred text representation of IPv6
   addresses is in one of three conventional forms.

   The preferred form is x:x:x:x:x:x:x:x, where the 'x's are the
   hexadecimal values of the eight 16-bit pieces of the address.

   Examples:

        FEDC:BA98:7654:3210:FEDC:BA98:7654:3210  (39 characters)

        1080:0:0:0:8:800:200C:417A  (25 characters)

   The second, or zero suppressed, form allows "::" to indicate multiple
   groups of suppressed zeroes, hence:

        1080:0:0:0:8:800:200C:417A

   may be represented as

        1080::8:800:200C:417A

   a saving of just 5 characters from this typical address form, and
   still leaving 21 characters.

   In other cases the saving is more dramatic, in the extreme case, the
   address:

        0:0:0:0:0:0:0:0

   that is, the unspecified address, can be written as

        ::

   This is just 2 characters, which is a considerable saving.  However
   such cases will rarely be encountered.


   The third possible form mixes the new IPv6 form with the old IPv4
   form, and is intended mostly for transition, when IPv4 addresses are
   embedded into IPv6 addresses.  These can be considerably longer than
   the longest normal IPv6 representation, and will eventually be phased
   out.  Consequently they will not be considered further here.

\item The New Encoding Format

   The new standard way of writing IPv6 addresses is to treat them as a
   128 bit integer, encode that in base 85 notation, then encode that
   using 85 ASCII characters.

\item Why 85?

   $2^{128}$ is 340282366920938463463374607431768211456.  $85^{20}$ is
   \\387595310845143558731231784820556640625, and thus in 20 digits of
   base 85 representation all possible $2^{128}$ IPv6 addresses can clearly
   be encoded.

   $84^{20}$ is 305904398238499908683087849324518834176, clearly not
   sufficient, 21 characters would be needed to encode using base 84,
   this wastage of notational space cannot be tolerated.

   On the other hand,
   \begin{equation} \label{bigEquation:1}
    94^{19} = 30862366077815087592879016454695419904 
   \end{equaltion}, also insufficient to encode
   all $2^{128}$ different IPv6 addresses, so 20 characters would be needed
   even with base 94 encoding.  As there are just 94 ASCII characters
   (excluding control characters, space, and del) base 94 is the largest
   reasonable value that can be used.  Even if space were allowed, base
   95 would still require 20 characters - equation \ref{bigEquation:1}.

   Thus, any value between 85 and 94 inclusive could reasonably be
   chosen.  Selecting 85 allows the use of the smallest possible subset
   of the ASCII characters, enabling more characters to be retained for
   other uses, eg, to delimit the address.

\item The Character Set

   The character set to encode the 85 base85 digits, is defined to be,
   in ascending order:

             '0'..'9', 'A'..'Z', 'a'..'z', '!', '#', '$', '%', '&', '(',
             ')', '*', '+', '-', ';', '<', '=', '>', '?', '@', '^', '_',
             '`', '{', '|', '}', and '~'.

   This set has been chosen with considerable care.  From the 94
   printable ASCII characters, the following nine were omitted:


      '"' and "'", which allow the representation of IPv6 addresses to
      be quoted in other environments where some of the characters in
      the chosen character set may, unquoted, have other meanings.

      ',' to allow lists of IPv6 addresses to conveniently be written,
      and '.' to allow an IPv6 address to end a sentence without
      requiring it to be quoted.

      '/' so IPv6 addresses can be written in standard CIDR
      address/length notation, and ':' because that causes problems when
      used in mail headers and URLs.

      '[' and ']', so those can be used to delimit IPv6 addresses when
      represented as text strings, as they often are for IPv4,

      And last, '\', because it is often difficult to represent in a way
      where it does not appear to be a quote character, including in the
      source of this document.

\item Converting an IPv6 address to base 85.

   The conversion process is a simple one of division, taking the
   remainders at each step, and dividing the quotient again, then
   reading up the page, as is done for any other base conversion.

   For example, consider the address shown above

        1080:0:0:0:8:800:200C:417A

   In decimal, considered as a 128 bit number, that is
   21932261930451111902915077091070067066.

   As we divide that successively by 85 the following remainders emerge:
   51, 34, 65, 57, 58, 0, 75, 53, 37, 4, 19, 61, 31, 63, 12, 66, 46, 70,
   68, 4.

   Thus in base85 the address is:

        4-68-70-46-66-12-63-31-61-19-4-37-53-75-0-58-57-65-34-51.

   Then, when encoded as specified above, this becomes:

        4)+k&C#VzJ4br>0wv%Yp

   This procedure is trivially reversed to produce the binary form of
   the address from textually encoded format.

\item Additional Benefit

   Apart from generally reducing the length of an IPv6 address when
   encode in a textual format, this scheme also has the benefit of
   returning IPv6 addresses to a fixed length representation, leading
   zeroes are never omitted, thus removing the ugly and awkward variable
   length representation that has previously been recommended.

\item Implementation Issues

   Many current processors do not find 128 bit integer arithmetic, as
   required for this technique, a trivial operation.  This is not
   considered a serious drawback in the representation, but a flaw of
   the processor designs.

   It may be expected that future processors will address this defect,
   quite possibly before any significant IPv6 deployment has been
   accomplished.

\item Security Considerations

   By encoding addresses in this form, it is less likely that a casual
   observer will be able to immediately detect the binary form of the
   address, and thus will find it harder to make immediate use of the
   address.  As IPv6 addresses are not intended to be learned by humans,
   one reason for which being that they are expected to alter in
   comparatively short timespan, by human perception, the somewhat
   challenging nature of the addresses is seen as a feature.

   Further, the appearance of the address, as if it may be random
   gibberish in a compressed file, makes it much harder to detect by a
   packet sniffer programmed to look for bypassing addresses. \cite{ipv6Authors}
\end{enumerate}

\begin{thebibliography}{12}

\bibitem{wikipediaRFC}
    Stephen D. Crocker, How the Internet Got Its Rules, The New York Times, 6 April 2009". The New York Times. April 7, 2009. Retrieved April 3, 2012.
\bibitem{networkingTruthsAuthors}
Network Working Group                                  R. Callon, Editor
Request for Comments: 1925                                          IOOF
Category: Informational                                     1 April 1996

\bibitem{theAddressIsTheMessageAuthor}
 S. Crocker
Request for Comments: 1776                               CyberCash, Inc.
Category: Informational                                     1 April 1995
\bibitem{coffeeAuthors}
Network Working Group                                       L. Masinter
Request for Comments: 2324                                 1 April 1998
\bibitem{UTF8UTF16Authors}
Network Working Group                                         M. Crispin
Request for Comments: 4042                             Panda Programming
Category: Informational                                     1 April 2005
\bibitem{ipv6Authors}
Network Working Group                                             R. Elz
Request for Comments: 1924                       University of Melbourne
Category: Informational                                     1 April 1996
\end{document}

